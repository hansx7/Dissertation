%%
% 摘要信息
% 本文档中前缀"c-"代表中文版字段, 前缀"e-"代表英文版字段
% 摘要内容应概括地反映出本论文的主要内容,主要说明本论文的研究目的、内容、方法、成果和结论。要突出本论文的创造性成果或新见解,不要与引言相 混淆。语言力求精练、准确,以 300—500 字为宜。
% 在摘要的下方另起一行,注明本文的关键词(3—5 个)。关键词是供检索用的主题词条,应采用能覆盖论文主要内容的通用技术词条(参照相应的技术术语 标准)。按词条的外延层次排列(外延大的排在前面)。摘要与关键词应在同一页。
% modifier: 黄俊杰(huangjj27, 349373001dc@gmail.com)
% update date: 2017-04-15
%%

\cabstract{
自动驾驶技术是近年来的一个研究热点。如何让车辆精确地感知周围环境是自动驾驶场景中的一个难题,而在传感器采集的一段连续数据中追踪道路上别的车辆就是环境感知的一部分。本文提出了一种融合两种开源框架的方法。该算法不仅对于每一帧数据都做独立的检测,还利用帧与帧之间场景变化较平稳的特点提取它们的相似性,从而优化检测结果,并达到更优的追踪效果。
}
\ckeywords{自动驾驶;激光雷达;深度学习;计算机视觉;点云;物体追踪}

\eabstract{
Automous driving technology is a research hotspot in recent years. Making a vehicle accurately perceive the surrounding environment is a difficult problem in automatic driving scenarios, and tracking other vehicles on the road from continuous data collected by sensors is a part of this. This paper proposes a fused method of two open source frameworks. It not only detects each frame data independently, but also extracts their similarities between adjacent frames, since the features of change stably, thus to optimize the detection results and achieve better tracking effect.
}
\ekeywords{Autonomous Driving, LiDAR(Light Detection And Ranging), Deep Learning, Computer Vision, Point Cloud, Object Tracking}

