%%
% 开题报告
% modifier: 黄俊杰(huangjj27, 349373001dc@gmail.com)
% update date: 2017-05-14

\usepackage{indentfirst}

% 选题目的
\objective{
\indent 自动驾驶是近年来的研究热点之一,拥有广阔的应用场景和强大的市场支持。自动驾驶的核心是环境感知,即让一辆车较为精确地感知周围环境。在驾驶环境中,车辆对行人的识别,对障碍物的检测和对路径的规划等都需要它能很好地理解环境。本项目基于激光雷达采集的三维点云数据集,采用深度学习的方法,目的是改善现有的物体追踪效果,从而为后续的决策提供可靠基础。
}

% 思路
\methodology{
\indent 激光雷达是自动驾驶领域应用最为广泛的传感器之一,它能采集到高分辨率的三位点云数据。深度学习为计算机视觉提供了重要的研究手段,它能在看似混杂的数据中提取出特征供机器来学习。因此本项目基于激光雷达采集到的三维点云数据集,并采用深度学习方法进行研究。
}
% 研究方法/程序/步骤
\researchProcedure{
\indent 本项目以一种物体检测效果显著的开源深度学习框架为基础,融合另一种物体追踪框架模型中的部分方法,进一步修改成为现在提出的Correlated-Voxelnet。
}

% 相关支持条件
\supportment{
实验室服务器:四路Tesla M40 GPU;
\newline 天河二号超算节点:tianhe2-G集群。
}

% 进度安排
\schedule{
第一阶段:2018年11月 - 2018年12月
\newline 论文调研,对目前已经被提出的研究方法建立全面的了解。
\newline 第二阶段:2019年1月 - 2019年2月
\newline 算法实现阶段。运用深度学习框架,结合多帧融合数据实现三维场景下的物体追踪。
\newline 第三阶段:2019年3月
\newline 代码调整和优化,以及论文写作。针对结果进行代码调整和参数优化,使得系统准确性和可靠性达到最优。同时撰写和修改论文。
}

% 指导老师意见
\proposalInstructions{

}

